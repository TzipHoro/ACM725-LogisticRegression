Statistical binary classification is a process used in medical testing to determine whether or not a patient has a certain disease.
It often involves supervised learning to determine decision classes based on predefined rules.
While methods such as decision trees, random forests, and neural networks are commonly used in other applications of binary classification, logistic regression offers medical testing a number of advantages due to its probabilistic nature and its relationship with the odds ratio~\citep{Schober2021-vs}. 
If interpreted correctly, it can provide a measure of associated risk between the independent variables and the binary outcome.

The upcoming sections will provide an overview of the statistical theory behind logistic regression, followed by an analysis of the association between various risk factors and the chances of a patient developing heart disease.
To measure the associated risk, this paper provides a synopsis of four common methods: \emph{risk difference}, \emph{relative risk}, \emph{odds ratios}, and \emph{marginal effects}, two of which are illustrated using a logistic regression model.