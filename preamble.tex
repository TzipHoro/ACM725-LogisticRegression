\usepackage{cite}
\usepackage{amsmath,amssymb,amsfonts}
\usepackage{algorithmic}
\usepackage{graphicx}
\usepackage{textcomp}
\usepackage{xcolor}
\def\BibTeX{{\rm B\kern-.05em{\sc i\kern-.025em b}\kern-.08em
    T\kern-.1667em\lower.7ex\hbox{E}\kern-.125emX}}

% additional packages
\usepackage{setspace}
\setstretch{2}
\usepackage[margin=1in]{geometry}
% \setlength\columnsep{25pt}
\hyphenpenalty=10000
\usepackage{natbib}
\usepackage{csquotes}
\usepackage{lipsum}

\usepackage{float}
\usepackage{adjustbox}
\usepackage{subcaption}
\usepackage[justification=centering]{caption}
\usepackage{wrapfig}
% \usepackage{dblfloatfix}
\usepackage{booktabs}
\usepackage{multirow}

\usepackage{url}
\def\UrlBreaks{\do\/\do-}
\usepackage{breakurl}

% risk factors
\newcommand{\ARD}[4]{\frac{#1}{#1 + #2} - \frac{#3}{#3 + #4}}
\newcommand{\RR}[4]{\frac{#1 / (#1 + #2)}{#3 / (#3 + #4)}}
\newcommand{\OR}[4]{\frac{#1 / #2}{#3 / #4}}

% math
\newcommand{\vec}[1]{\stackrel{\rightarrow}{#1}}
\newcommand{\matrx}[1]{\boldsymbol{#1}}
\newcommand{\argmax}[2]{\underset{#1}{\mathrm{argmax}} \left\{ #2 \right\}}
\newcommand{\norm}[1]{\left| #1 \right|}

% probability
\newcommand{\logit}[1]{\text{logit}\left( #1 \right)}
\newcommand{\prob}[1]{\mathbb{P} \left( #1 \right)}
\newcommand{\bern}[1]{\mathrm{Bernoulli} \left( #1 \right)}
\newcommand{\likelihood}[1]{\mathcal{L} \left( #1 \right)}
\newcommand{\clikelihood}[2]{\mathcal{L} \left( #1 | #2 \right)}

% limits
\newcommand{\Lim}[3]{\underset{#1 \to #2}{\text{lim}} #3}

% code blocks
\usepackage{listings}
\usepackage{xparse}

\definecolor{codegreen}{rgb}{0,0.6,0}
\definecolor{codegray}{rgb}{0.5,0.5,0.5}
\definecolor{codepurple}{rgb}{0.58,0,0.82}
\definecolor{backcolour}{rgb}{0.95,0.95,0.92}

\lstdefinestyle{code_block}{
    backgroundcolor=\color{backcolour},
    commentstyle=\color{codegreen},
    keywordstyle=\color{magenta},
    numberstyle=\tiny\color{codegray},
    stringstyle=\color{codepurple},
    basicstyle=\linespread{1}\ttfamily\footnotesize,
    breakatwhitespace=false,
    breaklines=true,
    captionpos=b,
    keepspaces=true,
    numbers=left,
    numbersep=5pt,
    showspaces=false,
    showstringspaces=false,
    showtabs=true,
    tabsize=4
}

\lstset{style=code_block}

\newcommand{\inlinecode}[2]{\lstinline[language=#1]$#2$}
\newcommand{\inlinecodettt}{\texttt}